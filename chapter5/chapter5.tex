\chapter{CONCLUSION}

Sequence alignment is an essential method in bioinformatics, serving as the foundation for many genomic analyses. Overlooking artifacts and errors during alignment reconstruction can impact downstream analyses, potentially leading to inaccurate findings in comparative and functional genomic studies. While such errors are eventually fixed in the reference genomes of model organisms through a laborious process, large amounts of genomic data used by researchers still contain these artifacts, often prompting the discarding of valuable data to prevent them from impacting results.

In this work, I designed, developed, and evaluated a novel statistical, codon-aware pairwise sequence aligner tailored to address common artifacts found in protein-coding genes. In chapter \ref{ch:alignpair}, I explained the aligner and its underlying evolutionary model, along with a comparative evaluation against conventional pairwise aligners, using human and gorilla homologous protein-coding sequences. Remarkably, despite humans and gorillas being two relatively close species, the results of my aligner demonstrated significant improvement. However, it is worth noting that the model exhibited a limiting computational runtime cost with sequences exceeding a few kilobases. Therefore, in chapter \ref{ch:marginal}, I presented and evaluated an approximate model that maintained similar accuracy while achieving competitive execution times. In the subsequent chapter \ref{ch:alndotplot}, I described a software package used for analyzing the alignment results from the previous chapter.

In the present era, the volume of genomic data generated is staggering, yet often remains unpolished due to the labor-intensive nature of genome curation. As a result, bioinformatic tools must continually evolve to incorporate models capable of addressing common artifacts. The ongoing development of software and statistical methodologies for biological sequence analysis will remain crucial, as they serve as the bridge transforming raw data into meaningful insights.
